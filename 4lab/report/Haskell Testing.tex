\documentclass[10pt,a4paper,final]{article} %draft

\usepackage[english, russian]{babel}

\usepackage{geometry}
%\usepackage[a4paper,left=2cm,right=2cm,top=2cm,bottom=2cm,bindingo ffset=0cm]{geometry}

\usepackage[T2A]{fontenc}
\usepackage[utf8]{inputenc}

\usepackage[utf8]{inputenc}

\usepackage[final]{pdfpages}

\usepackage{textcomp,enumitem}

\usepackage{amsmath,amsthm,amssymb}

\usepackage{longtable}
\usepackage{array}
\usepackage{adjustbox}

\usepackage{wrapfig}



\usepackage{listings}
\usepackage{xcolor}
\usepackage{caption}

\definecolor{codegray}{rgb}{0.95,0.95,0.95}
\definecolor{codepurple}{rgb}{0.58,0,0.82}
\definecolor{codegreen}{rgb}{0,0.5,0} % зеленый цвет
\definecolor{codeblue}{rgb}{0,0,0.5} % синий цвет
\definecolor{codestring}{rgb}{0.64,0.08,0.08} % цвет строк


\lstset{
	language=haskell,
	extendedchars=true,
	literate=
	{а}{{\selectfont\char224}}1
	{б}{{\selectfont\char225}}1
	{в}{{\selectfont\char226}}1
	{г}{{\selectfont\char227}}1
	{д}{{\selectfont\char228}}1
	{е}{{\selectfont\char229}}1
	{ж}{{\selectfont\char230}}1
	{з}{{\selectfont\char231}}1
	{и}{{\selectfont\char232}}1
	{й}{{\selectfont\char233}}1
	{к}{{\selectfont\char234}}1
	{л}{{\selectfont\char235}}1
	{м}{{\selectfont\char236}}1
	{н}{{\selectfont\char237}}1
	{о}{{\selectfont\char238}}1
	{п}{{\selectfont\char239}}1
	{р}{{\selectfont\char240}}1
	{с}{{\selectfont\char241}}1
	{т}{{\selectfont\char242}}1
	{у}{{\selectfont\char243}}1
	{ф}{{\selectfont\char244}}1
	{х}{{\selectfont\char245}}1
	{ц}{{\selectfont\char246}}1
	{ч}{{\selectfont\char247}}1
	{ш}{{\selectfont\char248}}1
	{щ}{{\selectfont\char249}}1
	{ъ}{{\selectfont\char250}}1
	{ы}{{\selectfont\char251}}1
	{ь}{{\selectfont\char252}}1
	{э}{{\selectfont\char253}}1
	{ю}{{\selectfont\char254}}1
	{я}{{\selectfont\char255}}1
	{А}{{\selectfont\char192}}1
	{Б}{{\selectfont\char193}}1
	{В}{{\selectfont\char194}}1
	{Г}{{\selectfont\char195}}1
	{Д}{{\selectfont\char196}}1
	{Е}{{\selectfont\char197}}1
	{Ж}{{\selectfont\char198}}1
	{З}{{\selectfont\char199}}1
	{И}{{\selectfont\char200}}1
	{Й}{{\selectfont\char201}}1
	{К}{{\selectfont\char202}}1
	{Л}{{\selectfont\char203}}1
	{М}{{\selectfont\char204}}1
	{Н}{{\selectfont\char205}}1
	{О}{{\selectfont\char206}}1
	{П}{{\selectfont\char207}}1
	{Р}{{\selectfont\char208}}1
	{С}{{\selectfont\char209}}1
	{Т}{{\selectfont\char210}}1
	{У}{{\selectfont\char211}}1
	{Ф}{{\selectfont\char212}}1
	{Х}{{\selectfont\char213}}1
	{Ц}{{\selectfont\char214}}1
	{Ч}{{\selectfont\char215}}1
	{Ш}{{\selectfont\char216}}1
	{Щ}{{\selectfont\char217}}1
	{Ъ}{{\selectfont\char218}}1
	{Ы}{{\selectfont\char219}}1
	{Ь}{{\selectfont\char220}}1
	{Э}{{\selectfont\char221}}1
	{Ю}{{\selectfont\char222}}1
	{Я}{{\selectfont\char223}}1,
	%	{|}{{\textbar}}1,
	%	{||}{{\textbar\textbar}}1
	%	{\&}{{\string&}}1
	%	{==}{{\string==}}1
	%	{\string\\}{{\string\\}}1,
	numbers=left,
	stepnumber=1,
	firstnumber=1,
	numberstyle=\tiny,
	basicstyle=\ttfamily\footnotesize,
	breakatwhitespace=false,
	breaklines=true,
	postbreak=\mbox{\textcolor{gray}{$\hookrightarrow$}}, % Символ переноса строки
	captionpos=b,
	keepspaces=true,
	showspaces=false,
	showstringspaces=false,
	showtabs=false,
	tabsize=2,
	frame=single,
	keywordstyle=\color{codeblue}, % ключевые слова
	commentstyle=\color{codegreen}, % комментарии
	stringstyle=\color{codestring}, % строки
	%identifierstyle=\color{green}, % стиль для переменных
	backgroundcolor=\color{codegray},
}
%\usepackage{fancyvrb}

%\usepackage{fancyhdr}

%\usepackage{upgreek}

%\usepackage{tipa}

%\usepackage{tikz}

%\usepackage{graphicx}

%\usepackage{pgfplots}

\usepackage{indentfirst}

%\usepackage{gensymb}

%\usepackage{amssymb}

%\usepackage{pdfpages}

\usepackage[unicode, pdftex, colorlinks, urlcolor=blue]{hyperref}

%\usepackage[T2A]{fontenc}


\usepackage{tabularx}
\usepackage{multirow}

\usepackage{booktabs} % для стильных линий таблицы



\usepackage{graphics}

\linespread{1.5}

\pagestyle{plain}

%\usepackage{listings} 
%\usepackage{moreverb} 

%\setlist[enumerate,itemize]{leftmargin=1.2cm} %отступ в перечислениях

\hypersetup{
	colorlinks=true,
	linkcolor=black,
	%allcolors=[RGB]{0 0 0}}
	}
	%\lstloadlanguages{ [LaTeX] TeX}
	
	\lstloadlanguages{ [LaTeX] TeX}
	
	
	
	\textheight=24cm 
	\textwidth=16cm
	\oddsidemargin=0pt 
	\topmargin=-1.5cm
	\parindent=24pt 
	\parskip=0pt 
	\tolerance=2000 
	\flushbottom 
	
	%\usepackage[font=scriptsize]{caption}
	\usepackage[labelsep=period]{caption}
	
	\begin{document}
\thispagestyle{empty}

\begin{center}
	{\Large МИНОБРНАУКИ РОССИИ}\\
	~\\
	{\large ФЕДЕРАЛЬНОЕ ГОСУДАРСТВЕННОЕ БЮДЖЕТНОЕ ОБРАЗОВАТЕЛЬНОЕ УЧРЕЖДЕНИЕ ВЫСШЕГО ПРОФЕССИОНАЛЬНОГО ОБРАЗОВАНИЯ}\\
	~\\
	{\Large \bf <<САНКТ-ПЕТЕРБУРГСКИЙ ПОЛИТЕХНИЧЕСКИЙ УНИВЕРСИТЕТ ПЕТРА ВЕЛИКОГО>>}\\
	~\\
	{\large Институт компьютерных наук и кибербезопасности}\\
	{\large Высшая школа технологий искусственного интеллекта}\\
	{\large Направление 02.03.01 Математика и компьютерные науки}\\
	~\\
	~\\
	~\\
	~\\
	{\Large \bf Отчет по практическому заданию №4}\\
	{\Large по дисциплине <<Функциональное программирование>> }\\
	~\\
	~\\
	
	%{\Large }\\
	%	{\Large \bf }\\ 	
	
	~\\
	~\\
	~\\
	~\\
	~\\
	~\\
	~\\
	~\\
	~\\
	{\large Обучающийся: \underline{\hspace{3.5cm}} \qquad\qquad Гладков И.А.}\\
	~\\
	{\large Руководитель: \underline{\hspace{3.5cm}} \hspace{15mm} Моторин Д.Е.}\\
	~\\
	~\\
\end{center}
\begin{flushright}
	
	«\underline{\hspace{1cm}}»\underline{\hspace{3cm}}20\underline{\hspace{0.7cm}}г.
\end{flushright}
~\\
\begin{center}
	{\large Санкт-Петербург, 2024}
\end{center}
\newpage

\tableofcontents

\newpage

\section* {Введение}
\addcontentsline{toc}{section}{Введение}
\par В данном практическом задании необходимо выполнить следующие задачи:
\begin{enumerate}
	\item Написать функцию логарифма \texttt{logBase :: Double -> Double -> Double}, которая принимает два числа: основание и аргумент, и возвращает логарифм аргумента по заданному основанию. Используя \texttt{QuickCheck}, проверьте следующие свойства:
	\begin{enumerate}
		\item Обратное свойство: \texttt{logBase b (b ** x) == х} для любого положительного \texttt{b} и \texttt{х}.
		
		\item Свойство смены основания: \texttt{logBase a b == logBase c b / logBase с а} для любых положительных \texttt{а, b и с}, где \texttt{с} не равно 1.
		
		\item Логарифм единицы: \texttt{logBase b 1 == 0} для любого положительного \texttt{b}.
	\end{enumerate}
	
	\item Написать функцию matrixMultiply :: Num a => [[a]] -> [[a]] -> [[a]], которая
	выполняет умножение двух матриц. Используя QuickCheck, проверьте следующие свойства:
	\begin{enumerate}
		\item Ассоциативность: \texttt{(А *B) * С == А * (В * С)} для любых матриц \texttt{А, В и С} (при условии, что размеры матриц позволяют это умножение).
		
		\item Дистрибутивность:\texttt{А * (B + C) == А*В + А * С} для любых матриц \texttt{А, В и С}.
		\item Умножение на единичную матрицу:\texttt{ А * 1 == А}, где \texttt{I} — единичная матрица соответствующего размера.
	\end{enumerate}
\end{enumerate}


Лабораторная работа выполнена на языке Haskell в текстовом редакторе Visual Studio Code 1.95.3.


\section{Математическое описание}

\textbf{QuickCheck} — это инструмент для автоматической проверки свойств программ. Он использует метод генерации тестов с последующей проверкой утверждений о программе с использованием случайных данных. QuickCheck позволяет описывать свойства функций в виде логических выражений, и затем автоматически проверять их с помощью случайных значений. 

В Haskell для работы с QuickCheck есть несколько основных команд и функций, которые позволяют генерировать тесты и проверять свойства:

\begin{itemize}
	\item \textbf{ quickCheck:} Команда для запуска проверки свойства. Она принимает функцию с описанием свойства и автоматически генерирует случайные входные данные для тестирования.
	\item\textbf{ Property:} Это тип, который представляет собой проверку логического свойства, которое может быть либо истинным, либо ложным, в зависимости от входных данных. Он часто используется с операторами, такими как ==>, чтобы описать свойства, которые должны выполняться при определённых условиях.
\end{itemize}

Для того чтобы QuickCheck мог генерировать случайные данные для тестов, необходимо определить, как создавать случайные значения для тех типов данных, которые используются в тестируемых функциях. Это делается через класс Arbitrary.
\par \textbf{Arbitrary} — это класс типов, который предоставляет метод arbitrary, используемый для генерации случайных значений. Для каждого типа, для которого мы хотим генерировать случайные значения, нужно предоставить инстанс этого класса.

\newpage
\section{Особенности реализации}
\subsection{Функция \texttt{logBase}}

\texttt{logBase} -- функция, принимающая два числа: основание и аргумент, и возвращающая логарифм аргумента по заданному основанию.
Функция представлена в листинге 1.

\begin{lstlisting}[caption = {Функция \texttt{logBase}}]
	logBase :: Double -> Double -> Double
	logBase a b = log b / log a
\end{lstlisting}





Для проверки равенства значений с плавающей точкой была создана функция  \texttt{almostEqual}, проверяющая их примерное равенство с заданной точностью.
\begin{lstlisting}
	almostEqual :: Double -> Double -> Double -> Bool
	almostEqual a b epsilon = abs (a - b) < epsilon
\end{lstlisting}

Далее были написаны тесты для данной функции в файле \texttt{Spec.hs}:

\begin{enumerate}
	\item Обратное свойство: \texttt{logBase b (b ** x) == х} для любого положительного \texttt{b} и \texttt{х}.
	
	\begin{lstlisting}
		prop_reverse :: Double -> Double -> Property
		prop_reverse b x = b > 0 && x > 0 && b /= 1 ==> almostEqual first second 1e-9
		where 
		first = logBase b (b ** x)
		second = x
	\end{lstlisting}
	
	\item Свойство смены основания: \texttt{logBase a b == logBase c b / logBase с а} для любых положительных \texttt{а, b и с}, где \texttt{с} не равно 1.
	
	\begin{lstlisting}
		prop_change :: Double -> Double -> Double -> Property
		prop_change a b c = a > 0 && b > 0 && c > 0 && c /= 1 && a /= 1 ==> almostEqual first second 1e-9
		where 
		first = logBase a b
		second = logBase c b / logBase c a
	\end{lstlisting}
	
	\item Логарифм единицы: \texttt{logBase b 1 == 0} для любого положительного \texttt{b}.
	
	\begin{lstlisting}
		prop_one :: Double -> Property
		prop_one b = b > 0 && b /= 1 ==> almostEqual first second 1e-9
		where 
		first = logBase b 1
		second = 0
	\end{lstlisting}
\end{enumerate}


\subsection{Функция \texttt{matrixMultiply}}

\texttt{matrixMultiply} -- функция, выполняющая умножение двух матриц. Код функции представлена в листинге 2.

\begin{lstlisting}[caption={Функция \texttt{matrixMultiply}}]
	matrixMultiply :: Num a => [[a]] -> [[a]] -> [[a]]
	matrixMultiply a b 
	| colsA /= rowsB = error "Incompatible matrix dimensions"
	| otherwise = map (\row -> map (sum . zipWith(*) row) (transpose b)) a
	where 
	colsA = length (head a)
	rowsB = length b
\end{lstlisting}

Она использует функция транспонирующую матрицу -- \texttt{transpose}
\begin{lstlisting}
	transpose :: [[a]] -> [[a]]
	transpose ([]:_) = []
	transpose x = map head x : transpose (map tail x)
\end{lstlisting}


Была написана функция создающая матрицу заданных размеров -- \texttt{genMatrix}

\begin{lstlisting}
	genMatrix :: (Arbitrary a) => Int -> Int -> Gen [[a]]
	genMatrix rows cols = replicateM rows (vectorOf cols arbitrary)
\end{lstlisting}

	Для проверки матриц, содержащих значения с плавающей точкой, используется функция проверяющая матрицы на примерное соответствие с заданной точностью.

\begin{lstlisting}
	almostEqualMatrices :: [[Double]] -> [[Double]] -> Double -> Bool
	almostEqualMatrices left right epsilon =
	if null left || null right then
	left == right 
	else
	let matrixLeft = concat left
	matrixRight = concat right
	in length matrixLeft == length matrixRight &&
	all (\(l, r) -> almostEqual l r epsilon) (zip matrixLeft matrixRight)
\end{lstlisting}

Далее были написаны тесты для данной функции в файле \texttt{Spec.hs}:

\begin{enumerate}

	\item Ассоциативность: \texttt{(А *B) * С == А * (В * С)} для любых матриц \texttt{А, В и С} (при условии, что размеры матриц позволяют это умножение).

	Для генерации корректных матриц, подходящих под условия проверки условия написана функция \texttt{genAssociativeMatrices}
	\begin{lstlisting}
		genAssociativeMatrices :: (Arbitrary a) => Gen ([[a]], [[a]], [[a]])
		genAssociativeMatrices = do
		m <- chooseInt (1, 5)  -- строки в матрице A
		n <- chooseInt (1, 5)  -- столбцы в A и строки в B
		p <- chooseInt (1, 5)  -- столбцы в B и строки в C
		q <- chooseInt (1, 5)  -- столбцы в C
		a <- genMatrix m n
		b <- genMatrix n p
		c <- genMatrix p q
		return (a, b, c)
	\end{lstlisting}
	Написаны тестирующие функции для проверки свойства перемножения матриц с целыми значениями  и с плавающей точкой
	
	\begin{lstlisting}
		prop_associativity_int :: Property
		prop_associativity_int =
		forAll (genAssociativeMatrices :: Gen ([[Int]], [[Int]], [[Int]])) $ \(a, b, c) ->
		matrixMultiply (matrixMultiply a b) c == matrixMultiply a (matrixMultiply b c)
		
		prop_associativity_double :: Property
		prop_associativity_double =
		forAll (genAssociativeMatrices :: Gen ([[Double]], [[Double]], [[Double]])) $ \(a, b, c) ->
		let left  = matrixMultiply (matrixMultiply a b) c
		right = matrixMultiply a (matrixMultiply b c)
		in almostEqualMatrices left right 1e-9
	\end{lstlisting}
	

	\item Дистрибутивность:\texttt{А * (B + C) == А*В + А * С} для любых матриц \texttt{А, В и С}.
	
	Для генерации корректных матриц, подходящих под условия проверки условия написана функция \texttt{genDistributiveMatrices}
	\begin{lstlisting}
	genDistributiveMatrices :: (Arbitrary a) => Gen ([[a]], [[a]], [[a]])
	genDistributiveMatrices = do
	m <- chooseInt (1, 5)  -- строки в матрице A
	n <- chooseInt (1, 5)  -- столбцы в A и строки в В и С
	q <- chooseInt (1, 5)  -- столбцы в В и столбцы в С
	a <- genMatrix m n
	b <- genMatrix n q
	c <- genMatrix n q
	return (a, b, c)
	\end{lstlisting}
	Написаны тестирующие функции для проверки свойства перемножения матриц с целыми значениями  и с плавающей точкой
	
	\begin{lstlisting}
prop_distributivity_int :: Property
prop_distributivity_int =
forAll (genDistributiveMatrices :: Gen ([[Int]], [[Int]], [[Int]])) $ \(a, b, c) ->
matrixMultiply a ( zipWith(zipWith(+)) b c) == zipWith(zipWith(+)) (matrixMultiply a b) (matrixMultiply a c)

prop_distributivity_double :: Property
prop_distributivity_double =
forAll (genDistributiveMatrices :: Gen ([[Double]], [[Double]], [[Double]])) $ \(a, b, c) ->
let left  = matrixMultiply a ( zipWith(zipWith(+)) b c)
right = zipWith(zipWith(+)) (matrixMultiply a b) (matrixMultiply a c)
in almostEqualMatrices left right 1e-9

	\end{lstlisting}
	
	
	\item Умножение на единичную матрицу:\texttt{ А * 1 == А}, где \texttt{I} — единичная матрица соответствующего размера.
	
	
	Для генерации корректных матриц, подходящих под условия проверки условия написаны функции \texttt{iMatrix}  и \texttt{genMultiplicationMatrices}
	\begin{lstlisting}
		iMatrix :: Num a => Int -> [[a]]
		iMatrix n = [[if i == j then 1 else 0 | j <- [1..n]] | i <- [1..n]]
				
		genMultiplicationMatrices :: (Arbitrary a, Num a) => Gen ([[a]], [[a]])
		genMultiplicationMatrices = do
		m <- chooseInt (1, 5)  -- строки в матрице A
		n <- chooseInt (1, 5)  -- столбцы в A и строки и столбцы в I 
		a <- genMatrix m n
		b <- return (iMatrix n)
		return (a, b)

	\end{lstlisting}
	Написаны тестирующие функции для проверки свойства перемножения матриц с целыми значениями  и с плавающей точкой
	
	\begin{lstlisting}
	prop_multiplication_int :: Property
	prop_multiplication_int =
	forAll (genMultiplicationMatrices :: Gen ([[Int]], [[Int]])) $ \(a, b) ->
	matrixMultiply a b == a
	
	prop_multiplication_double :: Property
	prop_multiplication_double =
	forAll (genMultiplicationMatrices :: Gen ([[Double]], [[Double]])) $ \(a, b) ->
	let left  = matrixMultiply a b
	right = a
	in almostEqualMatrices left right 1e-9
	\end{lstlisting}
	
	
\end{enumerate}

\subsection{Функция \texttt{main}}

Функция \texttt{main} является точкой входа программы, в которой выполняются тесты для проверки свойств различных функций. В частности, для каждого теста используется функция \texttt{quickCheck}, которая автоматически проверяет заданное свойство на случайных данных.

Функция представлена на листинге 3.

\begin{lstlisting}
	main :: IO ()
	main = do
	putStrLn("Тест prop_reverse:")
	quickCheck prop_reverse
	
	putStrLn("\nТест prop_change:")
	quickCheck prop_change
	
	putStrLn("\nТест prop_one:")
	quickCheck prop_one
	
	putStrLn("\n-----------------------------------")
	
	putStrLn("\nТест prop_associativity (with Int):")
	quickCheck prop_associativity_int 
	
	putStrLn("\nТест prop_associativity (with Double):")
	quickCheck prop_associativity_double
	
	putStrLn("\nТест prop_distributivity (with Int):")
	quickCheck prop_distributivity_int 
	
	putStrLn("\nТест prop_distributivity (with Double):")
	quickCheck prop_distributivity_double
	
	putStrLn("\nТест prop_multiplication (with Int):")
	quickCheck prop_multiplication_int 
	
	putStrLn("\nТест prop_multiplication (with Double):")
	quickCheck prop_multiplication_double
\end{lstlisting}

\newpage
\section{Результаты работы программы}

Результаты запуска программы с помощью команды \texttt{stack test} представлены ниже:
\begin{lstlisting} 
	Тест prop_reverse:
	+++ OK, passed 100 tests; 223 discarded.
	
	Тест prop_change:
	+++ OK, passed 100 tests; 423 discarded.
	
	Тест prop_one:
	+++ OK, passed 100 tests; 107 discarded.
	
	-----------------------------------
	
	Тест prop_associativity (with Int):
	+++ OK, passed 100 tests.
	
	Тест prop_associativity (with Double):
	+++ OK, passed 100 tests.
	
	Тест prop_distributivity (with Int):
	+++ OK, passed 100 tests.
	
	Тест prop_distributivity (with Double):
	+++ OK, passed 100 tests.
	
	Тест prop_multiplication (with Int):
	+++ OK, passed 100 tests.
	
	Тест prop_multiplication (with Double):
	+++ OK, passed 100 tests.
\end{lstlisting}



\newpage
\section* {Заключение}
\addcontentsline{toc}{section}{Заключение}


В ходе выполнения лабораторной работы были реализованы и протестированы на корректность свойства нескольких функций в языке программирования Haskell. У первой фунции, находящей логарифм числа по заданному основанию: обратное свойство, свойство смены основания,  логарифм единицы. У второй функции, перемножающей матрицы: ассоциативность, дистрибутивность, умножение на единичную матрицу.

В отчете приведены результаты тестирования каждого заданного свойства для обоих функций с использованием библиотеки QuickCheck.




\newpage
%\section*{Список литературы}
%\addcontentsline{toc}{section}{Список литературы} % Добавляем раздел в содержание

\begin{thebibliography}{99}
	
	\bibitem{kurtz2019}
	Курт У. \textit{Программируй на Haskell} / пер. с англ. С. Соловьева. — Москва: ДМК Пресс, 2019. — 384 с.
	

\end{thebibliography}

\addcontentsline{toc}{section}{Список литературы} % Добавляем раздел в содержание







\end{document}